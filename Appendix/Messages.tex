This appendix lists the different types of content (Table \ref{tab:MessageContent}) and performatives (Table \ref{tab:Messageperformative}) that can be put in a message. 


\begin{table}
    \centering
    \begin{tabular}[t]{| >{\raggedright\arraybackslash}m{0.3\linewidth} | >{\raggedright\arraybackslash}m{0.3\linewidth}  | >{\raggedright\arraybackslash}m{0.3\linewidth} | } 
         \hline
         Class & Types & Notations \\
         \hline
         \hline
         Elementary Values & Boolean, Integer, Double & b, i, d \\
         \hline
         Semiotic Components & Examples, Generalizations & e, g \\
         \hline
         Semiotic Elements & Extensional Definition, Sign, Intensional Definition & $s$, $E$, $I$, $s(C)$, $E(C)$, $I(C)$ \\ 
         \hline
         Arguments & Root Arguments, Counter-Arguments & $\alpha$, $\beta$, \ldots \\ 
         \hline
         Identifiers & Concept Identifiers, Argument Identifiers & $i$, $i(C)$, $i(\alpha)$ \\
         \hline
         Evaluations & R-Triplets, Pairing Relations & $r(C_{i}, C_{j}, U)$, $(i_{-1}, i_{0}, i_{1})$, $C_{i} r_{U} C_{j}$, $\equiv$, $\odot$, $\oslash$, $\otimes$ \\
         \hline
    \end{tabular}
    \caption{Different types of content available for messages. The different types are regrouped by classes in the left column. The right column gives some examples of content in their canonical notation.}
    \label{tab:MessageContent}
\end{table}

\begin{table}
    \centering
    \begin{tabular}[t]{| >{\raggedright\arraybackslash}m{0.2\linewidth} | >{\raggedright\arraybackslash}m{0.25\linewidth}  | >{\raggedright\arraybackslash}m{0.45\linewidth} | } 
         \hline
         Performative & Content & Description \\
        \hline
        \hline
        Accept-Root, Accept-Argument & $i(\alpha)$ & Tells $A_{-k}$ that the argument $\alpha$ has been accepted by $A_{k}$'s standards.\\
        \hline
        Assert & $s(C)$, $i(C,A_{k})$, $I(C)$ & Informs the agent $A_{-k}$ that $A_{k}$ has a concept $C$, and shares its sign and intensional definition with $A_{-k}$.\\
        \hline
        Baptize & $s$ & Attributes a new sign $s$ to the concept that is currently being created. \\
        \hline
        Debate & $i(C_{i})$, $i(C_{j})$ & Proposes to resolve the disagreement caused by the relation between $C_{i}$ and $C_{j}$.\\
        \hline
        Examples & $E$ & Contains a set of examples $E$ for $A_{-k}$ to expends its local context.\\
        \hline
        Evaluation & $i(C_{i})$, $i(C_{j})$, $r(C_{i}, C_{j}, U)$ & Shares the r-triplet of $C_{i}$ and $C_{j}$ with $A_{-k}$.\\
        \hline
        Intransitive & $i(C_{1}), \ldots, i(C_{n})$ & Informs the agent $A_{-k}$ that from the point of view of $A_{k}$, $C_{1} \ldots C_{n}$ are breaking the transitivity rule of the equivalence pairing relation.\\
        \hline
        Name & $s$, $e$ & Tells $A_{-k}$ that $A_{k}$ associates the example $e$ with the sign $s$.\\
        \hline
        Relation & $i(C_{i})$, $i(C_{j})$, $r$ & Shares the pairing relation between $C_{i}$ and $C_{j}$ with $A_{-k}$.\\
        \hline
        Remove & $i(C_{1})$, $\ldots$, $i(C_{n})$ & Tells $A_{-k}$ that all instances of the concepts $i(C_{1})$, $\ldots$, $i(C_{n})$ should be removed from the argumentation.\\
        \hline
        Replace & $s$, $i(C)$ & Tells $A_{-k}$ that the sign of concept $C$ is now $s$.\\
        \hline
        Root-Argument, Counter-Argument & $\alpha$ & Sends an argument $\alpha$ to $A_{-k}$.\\
        \hline
        Seize & & Notifies $A_{-k}$ that $A_{k}$ will take care of an asymmetrical aspect of the argumentation during the next turn(s).\\
        \hline
        Self-Check & & Asks $A_{-k}$ to look for the pairing relations $R(S_{K,-k}, S_{H,-k}, U_{O})$.\\
        \hline
        Size & $i$ & Contains the size of a set of examples. Used to determine if a new concept has potentially enough examples to deserved to be created.\\
        \hline
        Vote & $s$, $i(C)$, $d$ & Gives a support of value $d$ to the fact that the concept $C$ should have $s$ for sign.\\
        \hline
    \end{tabular}
    \caption{The different performatives available for messages. Each performative (left column) is presented with the type of content that is expected to be found in its instances (middle column). The role of each performative is presented in the right column.}
    \label{tab:Messageperformative}
\end{table}