% Introduction

\section{Introduction}

Binding the Introduction chapter with the plan of this state of the art.

\section{Knowledge Representation}

Presenting why knowledge representation is central in MAS, speak about heterogeneous data, the present challenges faced in Computer Sciences with Big Data, IoT, etc (see Paula)

\subsection{General Approaches of Knowledge Representation}

Really general introduction to knowledge representations, mostly to introduce the general dichotomy of grounder / connected meaning that we will find everywhere else

\subsubsection{Dictionary Approach}

Self explanatory. Since we are not focusing on this approach, we explain its limitations on the stuff that we want to do later: consistency of vocabularies, minimal number of concepts etc.

\subsubsection{Encyclopedic Approach}

Again, general presentation. Already linking it to the Connectivist approach on meaning.

\subsection{Knowledge Representation and Machine Learning}

Mainly to introduce the vocabulary and formalism used later in the text.

\subsubsection{Case-Based Reasoning}

Presenting CBR here, with opening on argumentation theory and work done with AMAIL.

\subsubsection{Symbol Grounding}

How to represent grounded meanings, FCG, etc

\subsubsection{Connectivism}

Continue the Encyclopedic approach in the introduction of contrast sets and related meanings

\subsection{Ontological Knowledge}

Still connected meaning, but more formalized and organized. Talk about the distinction between classes and categories, object oriented approach etc synthesis of encyclopedic and conectivist approach

\subsection{Cognitive Science}

The philosophical ground of out work

\subsubsection{Philosophy}

Wittgenstein, Pierce as situated meaning (language is its use)

\subsubsection{Ethology}

Presentation of Frake and the contrast set, relation with Ontology, Encyclopedic knowledge and connectivist approach of meaning, the contrast set as a unit of category and the concept as the unit of class.

\section{Communication}

Now that knowledge representation has been introduced, how agents can communicate about them

\subsection{Language Philosophy}

Introducing Saussure, linked to the connectivists, then speaking about semiotics but still introducing more general notions

\subsection{Information Theory}

Introducing Shannon and more generally computational take on communication, Pierce's adaptation and criticts (no context), the problem of semantic noise and how it is consistent with the notion of contrast sets

\section{Binding Knowledge Representation and Communication}

Restating the problematic of joining the constraints on KR and communication for agents.

\subsection{Ontology Matching}

Matching ontologies, problem: suppose an alignment on classes if not labels

\subsection{Argumentation Theory}

Arguing on concepts, problem: suppose an alignement of labels then match classes

\subsection{Semiotics}

How the triadic relations allows to map unmatching classes and labels


MOSTLY, THE JOB IS TO INTRODUCE A DISAMBIGUATION BETWEEN CONCEPTS LIKE "CLASSES \& CONCEPTS" OR "LABEL, SIGN, SYMBOL" etc

Agents are by definition taking actions in their environment. This requires that they are able to classify their stimuli (elements from their environment), their actions and the consequences of their action. This is often formalized as the input/output approach in ML. So Agents can be seen as classifiers.

Agents that use language are able to communicate with other agents on these classifications. They associate each class with a sign. Then they use the sign to access the different speech acts that allows language, but not our focus here. Problem: how to guarantee that the class-sign associations of two different agents are consistent, and what ``consistent'' means here?

First, how these classes are organized. They are organized in different category (if a class is a color, then a category is 'colors'). Explanation of how to build categories upon classes is given by -- give a synthesis of the state of the art -- and we pick the explanation of ethnography: the contrast set.

The contrast set gives a connexionist approach of the organization of categories, which is okay to define the interactions of classes between one category, but does not explain how to versions of a category belonging to two different agents can be consistent. For this, we take a look to semiotics, the science of signs and their relation with knowledge and reality. 

The present thesis focuses on a situation where two agents that {\color{red} learned their knowledge under different supervised learning? Or do we want to be more general? I remember that you didn't want to talk about learning. However, wouldn't it be helpful to present later the expected differences between the agent's knowledge that we formalize as contrast sets?}

% Communication

% Contrast sets

% Concepts

% Mutual Intelligibility

% Semantic alignement

% Ontology mapping

% Argumentation on the meaning