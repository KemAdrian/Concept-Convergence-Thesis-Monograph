% Idea of discussion about the application of our PhD in a multi-level ontology
The main limitation of the thesis is that it does not present any grounding to the symbols: an argumentation to reach a mutual intelligibility over a given contrast set requires a prior agreement over the formulation language of our logical propositions that represent our examples and our generalizations (here, the feature term formalism).

% Problem with the recursivity of meaning
Each sort of this formalism can be seen as a particular contrast set, therefore we just moved the alignment problem one level lower in our ontology. Even by changing our formalism, we wouldn't be able to get rid of that problem -- because as long as we consider the language as a closed system of signs we encounter the "Wittgenstein" problem that everything is just a language game between our agents and there is no fix point, no base contrast set, that we can align without the a-priori of another contrast set alignment. This is the grounding problem.

% The meaning not as a classification but as a causation
However, it is possible to bypass this problem if we consider that what grounds the meaning is not a classification that should be shared by all agents (we would still fall in a recursive approach that leads to a dead end), but a measure of causation instead.